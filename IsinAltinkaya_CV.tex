
%-------------------------
% Author : Isin Altinkaya
% Website: https://github.com/isinaltinkaya/cv
%------------------------

\documentclass[letterpaper,10.5pt]{article}
\usepackage{latexsym}
\usepackage[empty]{fullpage}
\usepackage{titlesec}
\usepackage{marvosym}
\usepackage[usenames,dvipsnames]{color}
\usepackage{verbatim}
\usepackage{enumitem}
\usepackage{fancyhdr}
\usepackage{graphicx}
\usepackage{bold-extra}
\usepackage[utf8]{inputenc}  % input encoding
\usepackage{mathtools} % align text

%icons
\usepackage{orcidlink}


\newcommand\sbullet[1][.5]{\mathbin{\vcenter{\hbox{\scalebox{#1}{$\bullet$}}}}}


\usepackage{xcolor}
\definecolor{isored}{HTML}{980000}


% \newcommand\ytl[2]{
% \parbox[b]{8em}{\hfill{\color{cyan}\bfseries\sffamily #1}~$\cdots\cdots$~}\makebox[0pt][c]{$\bullet$}\vrule\quad \parbox[c]{4.5cm}{\vspace{7pt}\color{red!40!black!80}\raggedright\sffamily #2.\\[7pt]}\\[-3pt]}
% \begin{table}
% \ytl{1947}{AT and T Bell Labs develop the idea of cellular phones}
% \ytl{1968}{Xerox Palo Alto Research Centre envisage the `Dynabook'}
% \ytl{1971}{Busicom 'Handy-LE' Calculator}
% \ytl{1973}{First mobile handset invented by Martin Cooper}
% \ytl{1978}{Parker Bros. Merlin Computer Toy}
% \ytl{1981}{Osborne 1 Portable Computer}
% \ytl{1982}{Grid Compass 1100 Clamshell Laptop}
% \ytl{1983}{TRS-80 Model 100 Portable PC}
% \ytl{1984}{Psion Organiser Handheld Computer}
% \ytl{1991}{Psion Series 3 Minicomputer}
% \end{table}


\usepackage[pdftex]{hyperref}

\hypersetup{
    colorlinks = true,
    linkcolor = blue,
    anchorcolor = blue,
    citecolor = blue,
    filecolor = blue,
    urlcolor = blue,
}

    
% Adjust margins
\addtolength{\oddsidemargin}{-0.375in}
\addtolength{\evensidemargin}{-0.375in}
\addtolength{\textwidth}{1in}
\addtolength{\topmargin}{-.5in}
\addtolength{\textheight}{1.0in}
\urlstyle{same}
\raggedbottom
\raggedright
\setlength{\tabcolsep}{0in}

% Sections formatting
\titleformat{\section}{
  \vspace{-4pt}\bfseries\scshape\raggedright\large\color{isored}
}{}{1em}{}[\color{isored}{\titlerule} \vspace{-3pt}]

% Custom commands
\newcommand{\cvItem}[2]{
  \item\small{
    \textbf{#1}{: #2 \vspace{-2pt}}
  }
}

\newcommand{\cvSubheading}[5]{
  \vspace{-1pt}\item
    \begin{tabular*}{0.97\textwidth}{l@{\extracolsep{\fill}}r}
      \textbf{#1} & #2 \\
      {\small#3} & {\small #4} \\
    \end{tabular*}\vspace{3pt} \\
    #5
}

\newcommand{\cvSubItem}[2]{\cvItem{#1}{#2}\vspace{-4pt}}

%\renewcommand{\labelitemii}{$\circ$}
%\renewcommand{\labelitemii}{$\rightarrow$}


\newcommand{\cvSubHeadingListStart}{\begin{itemize}[leftmargin=*]}
\newcommand{\cvSubHeadingListEnd}{\end{itemize}}
\newcommand{\cvItemListStart}{\begin{itemize}}
\newcommand{\cvItemListEnd}{\end{itemize}\vspace{-5pt}}
%
%\newcommand\textline[4][t]{%
%  \par\smallskip\noindent\parbox[#1]{.333\textwidth}{\raggedright\texttt#2}%
%  \parbox[#1]{.333\textwidth}{\centering#3}%
%  \parbox[#1]{.333\textwidth}{\raggedleft\texttt{#4}}\par\smallskip%
%}

%-------------------------------------------
%%%%%%  CV STARTS HERE  %%%%%%%%%%%%%%%%%%%%

\begin{document}

%----------HEADING-----------------

\href{http://isinaltinkaya.com/}{isinaltinkaya.com} 
\color{isored}
\hfill \begin{Huge} Isin Altinkaya \end{Huge} \\
\color{black}

\href {https://github.com/isinaltinkaya}{GitHub/isinaltinkaya} 
\hfill Oester Voldgade 5-7, 1350 \\

\href {https://x.com/IsinAltinkaya}{X(Twitter)/isinaltinkaya}
\hfill Copenhagen K, Denmark \\

\href{mailto:isinaltinkaya@gmail.com}{isinaltinkaya@gmail.com} $\sbullet[.5]$ \href{mailto:isin.altinkaya@sund.ku.dk}{isin.altinkaya@sund.ku.dk} 
\hfill \href{https://scholar.google.com/citations?user=DHpwS6oAAAAJ&hl=en&oi=ao}{Google Scholar} $\sbullet[.5]$ \href{https://orcid.org/0000-0002-6364-3332}{ORCID}\orcidlink{0000-0002-6364-3332} $\sbullet[.5]$ 
\href{https://www.linkedin.com/in/altinkayaisin/}{LinkedIn} \\



%I\c{s}{\i}n Alt{\i}nkaya

%-----------EDUCATION-----------------


\section{Education}
  \cvSubHeadingListStart
    
    %%
    \cvSubheading
      {PhD} {University of Copenhagen, Denmark}
      {Faculty of Health and Medical Sciences, Globe Institute, Section for GeoGenetics} {2022/1 - 2025/3 (expected)}
    
        \quad \textbf{Title:} Statistical Methods for Population Genetic Inference Based on Low-Depth Sequencing Data \\
        \quad \textbf{Advisors:} Thorfinn Sand Korneliussen and Rasmus Nielsen \\

    %%
    \cvSubheading
      {BSc in Biology} {Hacettepe University, Turkey}
      {Faculty of Science, Department of Biology} {2015/8 - 2020/7}
  \cvSubHeadingListEnd

%    \quad GPA: 3.23, \quad Certificate No: 20-321-032

%-----------RESARCH EXPERIENCE-----------------
\section{Research Experience}
    \cvSubHeadingListStart

         \cvSubheading
        {Visiting Researcher at the Moltke Lab}
        {University of Copenhagen, Denmark}
        {Faculty of Science, Department of Biology, Section for Computational and RNA Biology}
        {2024/09 - 2024/12}
          \vspace{-0.1em} \begin{enumerate}[label=\textbf{-},nosep,wide,  labelindent=0pt]
        \item Collaborated with Ida Moltke on a research project on IBD segment identification with low-depth sequencing data.
    \end{enumerate}

        \cvSubheading
        {PhD Fellow}
        {University of Copenhagen, Denmark}
        {Faculty of Health and Medical Sciences, Globe Institute, Section for GeoGenetics}
        {2022/1 - Present (expected 2025/3)}
            \vspace{-0.1em} \begin{enumerate}[label=\textbf{-},nosep,wide,  labelindent=0pt]
        \item Supervised by Thorfinn Sand Korneliussen and Rasmus Nielsen.
        \item Developed statistical methods and frameworks for population genetic inference, with a focus on low-depth NGS data.
        \item Implemented computational tools for genomic analysis emphasizing efficiency and accuracy.
        \item Authored three first-author papers: one under review and two manuscripts in preparation.
        \item Presented research findings at three conferences and one poster session.
        \item Contributed to collaborative projects outside of the primary PhD projects, resulting in one paper published in \textit{Nature}, and two papers submitted to \textit{Nature} (under review).
        \item Delivered training on workload management using Slurm in the \textit{Introduction to Slurm} workshop for the section.
        \item Gained and actively applied expertise in C, C++, Python, and R programming, Git, Conda/Bioconda, Snakemake, Make, \LaTeX, sed, AWK, and Bash, as well as utilizing Slurm and HPC environments for efficient execution of large-scale research and computational tasks.

      \end{enumerate}
    \medskip
  
        \cvSubheading
        {Research Assistant}
        {University of Copenhagen, Denmark}
        {Faculty of Health and Medical Sciences, Globe Institute, Section for GeoGenetics}
        {2020/10 - 2021/12}
          \vspace{-0.1em} \begin{enumerate}[label=\textbf{-},nosep,wide,  labelindent=0pt]
        \item Contributed to developing and evaluating the Lundbeck Center's ancient DNA bioinformatics pipeline, facilitating the processing of hundreds of ancient genomes, resulting in one preprint.
        \item Contributed to the development and maintenance of the \href{https://github.com/ANGSD/angsd/}{ANGSD} toolkit.
        \item Contributed to a collaborative research project, resulting in a publication.
        \item Managed the Geogenetics servers used by many research groups as a server administrator, supporting large-scale computational research and bioinformatics workflows.
        \item Utilized skills in diverse research and computational tasks, including sequence alignment, performing analyses such as PCA and admixture inference, and developing scalable pipelines for high-throughput genomic data processing.
    \end{enumerate}

    \medskip
    
    \cvSubheading
      {Research Intern at the Fumagalli Lab}
      {Imperial College London, United Kingdom}
      {Department of Life Sciences}
      {2019/7 - 2019/9}
          \vspace{-0.1em} \begin{enumerate}[label=\textbf{-},nosep,wide,  labelindent=0pt]
        \item Supervised by Matteo Fumagalli.
        \item Contributed to the HMMPloidy project by implementing methods such as genotype likelihood estimation for arbitrary N-ploidies, resulting in a publication.
        \item Supported by the Erasmus+ Traineeship Grant.
        \item Enhanced proficiency in Python programming, Bash/Shell scripting, and Git.
    \end{enumerate}
    \medskip

    \cvSubheading
      {Research Intern at the Biogeography Research Lab}
      {Hacettepe University, Turkey}
      {Faculty of Science, Department of Biology}
      {2019/2 - 2020/7}
          \vspace{-0.1em} \begin{enumerate}[label=\textbf{-},nosep,wide,  labelindent=0pt]
        \item Supervised by Bar{\i}\c{s} \"{O}z\"{u}do\u{g}ru.
        \item Developed and optimized a Snakemake pipeline for phylogenomic analyses using RADSeq data, resulting in a manuscript in preparation.
        \item Delivered presentations on Linux/Bash and phylogenetic methods during internal lab meetings and taught the practical session in the \textit{Introduction to Bioinformatics and Computational Biology} workshop, refining teaching and presentation skills.
        \item Enhanced proficiency in Python, Snakemake, R, Bash/Shell scripting, Git, and gained experience in executing large analyses on HPC clusters with Slurm.
    \end{enumerate}
    \medskip
 
    \cvSubheading
    {Research Intern at the Comparative and Evolutionary Biology Lab}
    {Middle East Technical University, Turkey}
    {Department of Biological Sciences: Molecular Biology and Genetics}
    {2018/8 - 2019/6}
          \vspace{-0.1em} \begin{enumerate}[label=\textbf{-},nosep,wide,  labelindent=0pt]
        \item Supervised by Mehmet Somel.
        \item Performed analyses to evaluate the effects of the properties of probe sequences in the 1240K SNP capture method.
        \item Applied and further developed skills in data analysis using R.

      \end{enumerate}
    \medskip

    \cvSubheading
      {Research Intern at the Evolutionary and Quantitative Genetics Lab}
      {Hacettepe University, Turkey}
     {Faculty of Science, Department of Biology}
      {2015/10 - 2018/12}
          \vspace{-0.1em} \begin{enumerate}[label=\textbf{-},nosep,wide,  labelindent=0pt]
        \item Supervised by Ergi Deniz \"{O}zsoy.
        \item Conducted statistical analyses for quantitative genetics projects using Drosophila Genetic Reference Panel 2 data, which resulted in a poster presentation and acknowledgments in two MSc and one PhD thesis.
        \item Gained and applied skills in data analysis using R.
    \end{enumerate}
  \cvSubHeadingListEnd

%----------- PUBLICATIONS------------
\section{Publications}

\begin{flalign*}
\href{https://www.biorxiv.org/content/10.1101/2024.12.02.626332v1}{\textbf{[6]}} &\; \text{Fulya E. Yediay [et al, including \textbf{Isin Altinkaya}]. Ancient genomics support deep divergence between} && \\
&\; \text{Eastern and Western Mediterranean Indo-European languages. (under review by \textit{Nature}). \textit{bioRxiv} (2024).} && \\
\href{https://doi.org/10.1101/2024.04.09.586324 }{\textbf{[5]}} &\; \text{\textbf{Isin Altinkaya}, Thorfinn S. Korneliussen, Rasmus Nielsen. vcfgl: A flexible genotype likelihood simulator} && \\ 
&\; \text{for VCF/BCF files. (under review by \textit{Bioinformatics}). \textit{bioRxiv} (2024).} \\ 
\href{https://doi.org/10.1101/2024.03.13.584607}{\textbf{[4]}} &\; \text{Hugh McColl [et al, including \textbf{Isin Altinkaya}]. Steppe Ancestry in western Eurasia and the spread of the Germanic} \\ 
&\; \text{Languages. \textit{bioRxiv} (2024).} && \\
\href{https://doi.org/10.1038/s41586-023-06865-0}{\textbf{[3]}} &\;  \text{Morten E. Allentoft [et al, including \textbf{Isin Altinkaya}]. {Population genomics of post-glacial western Eurasia}. \textit{Nature}} \\ 
&\; \text{(2024).} && \\
\href{https://doi.org/10.24072/pcjournal.178}{\textbf{[2]}} &\; \text{Samuele Soraggi, Johanna Rhodes, \textbf{Isin Altinkaya}, Oliver Tarrant, François Balloux, Matthew C. Fisher, Matteo} && \\
&\; \text{Fumagalli. HMMploidy: inference of ploidy levels from short-read sequencing data. \textit{Peer Community Journal} (2022).} && \\ 
\href{https://doi.org/10.1093/gigascience/giac032}{\textbf{[1]}} &\; \text{Alex Mas-Sandoval, Nathaniel S. Pope, Knud Nor Nielsen, \textbf{Isin Altinkaya}, Matteo Fumagalli, and Thorfinn S.} && \\
&\; \text{Korneliussen. Fast and accurate estimation of multidimensional site frequency spectra from low-coverage} && \\
&\; \text{high-throughput sequencing data. \textit{GigaScience} (2022).} 
\end{flalign*}

\smallskip \vspace{-0.2em}

\textbf{Manuscripts in Preparation}\vspace{-0.5em}
\begin{flalign*}
\textbf{[3]} &\; \text{\textbf{Isin Altinkaya}, Lei Zhao, Rasmus Nielsen, Thorfinn S. Korneliussen. ngsAMOVA: A genotype likelihood framework} && \\ 
&\; \text{for analysis of molecular variance (AMOVA) with low-depth sequencing data.} && \\
\textbf{[2]} &\; \text{\textbf{Isin Altinkaya}, Ida Moltke, Rasmus Nielsen, Thorfinn S. Korneliussen. IBDGL: A method for the accurate detection} && \\ 
&\; \text{of IBD segments in low- and medium-depth genomic data without phasing.} && \\
\textbf{[1]} &\; \text{\textbf{Isin Altinkaya}, Emrullah Y{\i}lmaz, {\.{I}}smail K. Sa{\u{g}}lam, Bar{\i}\c{s} \"{O}z\"{u}do\u{g}ru. Understanding the phylogenetic relationships} && \\ 
&\; \text{within the \textit{Noccaea} species complex from RADSeq data using the multi-species coalescent.} && \\ 
\end{flalign*} \vspace{-3em}


%-----------PROFESSIONAL EXPERIENCE-----------------
\section{Professional Experience}
  \cvSubHeadingListStart
    \cvSubheading
      {Freelance Programmer}{Remote}
      {Self-Employed}{2019/1 - 2020/10}
      \vspace{-0.2em}
      \begin{enumerate}[label=\textbf{-},nosep,wide, labelindent=0pt]
        \item Applied various \textbf{AI} and \textbf{ML} techniques such as \textbf{Random Forest} and \textbf{K-Means Clustering} to solve complex problems.
        \item Utilized programming languages and tools such as \textbf{Python}, \textbf{R}, \textbf{AWK}, and \textbf{Bash scripting} to perform data analysis.
        \item Delivered customized solutions for clients, developing strong problem-solving and programming skills.
      \end{enumerate}
  \cvSubHeadingListEnd


%--------SKILLS------------
\section{Skills}
\textbf{Programming languages}: C (advanced), C++ (intermediate), R (advanced), Python (advanced), Java (advanced beginner), Perl (advanced beginner), Rust (advanced beginner) \\
\textbf{Scripting and typesetting}: \LaTeX, Markdown, HTML\&CSS, JavaScript, Google Apps Script, sed, AWK, Bash \\
\textbf{Operating systems}: GNU/Linux (including server administration and HPC environments) \\
\textbf{Other tools and frameworks}: Git,  Conda/Bioconda, Snakemake, RShiny, Dash, Django, Make \\


% %-----------SELECTED COURSES-----------------
% \section{Relevant Courses}

% \textbf{Computational Statistics}, University of Copenhagen, Department of Mathematical Sciences, (2022) \\
% \textbf{Advanced Topics in Data Analysis}, University of Copenhagen, (2021) \\
% \textbf{Fundamentals in Computational Analysis of Large-Scale Datasets}, University of Copenhagen, (2021) \\
% \textbf{Stay-at-Home RevBayes Workshop}, Topics: Tree inference from molecular data, evaluating MCMC performance, assessing model adequacy and macroevolutionary analyses, (2020) \\
% \textbf{Fundamentals of Bioinformatics}, Department of Computer Science, Hacettepe University, Lect.: Tunca Do\u{g}an, (2019) \\
% \textbf{Population Genetics}, Graduate course, Department of Molecular Biology and Genetics, Middle East Technical University, Lect.: Mehmet Somel, (2018) \\
% \textbf{Population Genetics Simulations with R}, Middle East Technical University, (2018) \\
% \textbf{Introduction to Game Theory}, Anadolu University, (2017) \\
% \textbf{Evolutionary Genomics Winter School}, Hacettepe University, (2017) \\


%-----------PRESENTATIONS-----------------
\section{Conferences, Workshops, and Talks}

\textbf{Oral Presentations}\vspace{-0.5em}
\begin{flalign*}
\textbf{[6]} &\; \text{A Genotype Likelihood Framework for Identifying Identity-by-Descent (IBD) Segments Based on Low-Depth Sequencing} && \\
&\; \text{Data, \textit{Evolution and Population Genetics in Denmark (EPIC) Conference}, Denmark, (2023).} && \\
\textbf{[5]} &\; \text{A Genotype Likelihood Framework for the Analysis of Molecular Variance (AMOVA), \textit{Evolution and Population}} && \\
&\; \text{\textit{Genetics in Denmark (EPIC) Conference}, Denmark, (2022).} && \\
\textbf{[4]} &\; \text{Biological Evolution, \textit{Science and Future Magazine Science for Youth Seminar}, Turkey, (2018).} && \\
\textbf{[3]} &\; \text{Evolutionary Biology and Preventing the Species Extinction, \textit{7th National Environment and Ecology Student Congress},} && \\
&\; \text{Turkey, (2016).} && \\
\textbf{[2]} &\; \text{Evolutionary Biology and Environmental Problems, \textit{Turkey Meets Evolution: Bilkent University}, Turkey, (2016).} && \\
\textbf{[1]} &\; \text{Evolutionary Biology and Preventing the Species Extinction, \textit{7th National Environment and Ecology Student Congress},} && \\
&\; \text{Turkey, (2016).} && \\
\end{flalign*}

%-----------POSTER PRESENTATIONS-----------------
\vspace{-2em}
\textbf{Poster Presentations}\vspace{-0.5em}
\begin{flalign*}
\textbf{[2]} &\; \text{\textbf{Isin Altinkaya}, Lei Zhao, Rasmus Nielsen, Thorfinn S. Korneliussen. A Genotype Likelihood Framework for} && \\
&\; \text{Analysis of Molecular Variance, $d_{xy}$, and Neighbor-Joining Trees with Low-Depth Sequencing Data. \textit{Society for}} && \\
&\; \text{\textit{Molecular Biology and Evolution (SMBE)}, Italy, (2023).} && \\
\textbf{[1]} &\; \text{Damla Aygün, \textbf{Isin Altinkaya}, Murat Y{\i}lmaz, Ergi Deniz \"{O}zsoy, Efe Sezgin. Host Genetics of Microbiota} && \\
&\; \text{Diversity in \textit{Drosophila melanogaster}. \textit{Ecology and Evolutionary Biology Symposium (EEBST)}, Turkey, (2018).} && \\
\end{flalign*}
\vspace{-4em}


%-----------TEACHING EXPERIENCE------------
\section{Teaching Experience}
\vspace{-0.8em}
\begin{flalign*}
\textbf{-} &\; \text{Advanced Bioinformatics for NGS, Graduate Course, \textbf{Teaching Assistant}, University of Copenhagen, (2021).} && \\
\textbf{-} &\; \text{Introduction to Computational Biology and Bioinformatics, \textbf{Teaching Assistant}, Hacettepe University, (2020).} && \\
\textbf{-} &\; \text{Biometry, 4th Year Elective Course, \textbf{Teaching Assistant}, Hacettepe University, (2019).} && \\
\textbf{-} &\; \text{Introductory Evolutionary Biology, Science and Utopia Magazine Evolution Courses, \textbf{Lecturer}, (2017).} && \\
\textbf{-} &\; \text{Introduction to Evolution, Hacettepe University Evolution Workshops, \textbf{Lecturer}, (2016).} && \\
\end{flalign*} \vspace{-4em}

%-----------SELECTED SYMPOSIUM AND EVENTS------------
%\section{Other Academic Achievements, Honors, and Activities}
\section{Scientific Outreach, Public Engagement and Activities}

Contributed to the organization of various free symposiums, workshops, and events focusing on teaching evolution and scientific thinking to the public, with a special focus on underdeveloped regions. \\

\smallskip

\textbf{-}  12th Aykut Kence Evolution Conference, \textbf{Organizer}, (2018). \\

\textbf{-}  Science and Utopia Magazine, \textbf{Popular science writer}, Subject: Evolutionary biology, (2017/6 - 2018/5). \\

\textbf{-}  Popular Science Magazine, Turkey branch, \textbf{Popular science news editor and translator}, (2016/2 - 2016/5). \\

\textbf{-}  \href{https://evrimagaci.org/isinaltinkaya/}{Tree of Evolution}, \textbf{Writer and editor}; Subject: Evolutionary biology, (2015/9 - present).\\
Authored multiple popular science articles, which can be found at \href{https://evrimagaci.org/isinaltinkaya/}{evrimagaci.org/isinaltinkaya}.


\textbf{-}  Turkey Meets Evolution: Izmir, \textbf{Organizer}, (2012). \\

\textbf{-}  Tree of Evolution Bornova Anatolian High School Evolution Conference, \textbf{Organizer}, (2012). \\

\medskip

%-----------POPULAR PUBLICATIONS------------
\textbf{Popular Science Publications in Magazines (in Turkish):}
\setlength{\abovedisplayskip}{0pt}
\setlength{\belowdisplayskip}{0pt}
\setlength{\abovedisplayshortskip}{0pt}
\setlength{\belowdisplayshortskip}{0pt}
\begin{flalign*}
\textbf{[5]} &\; \text{\textbf{Isin Altinkaya} and Dr. Martin Hanczyc. Examining the thin line between living and non-living matter. \textit{Science and}} && \\
&\; \text{Utopia Magazine, \textit{Issue: 283} (2018).} && \\
\textbf{[4]} &\; \text{\textbf{Isin Altinkaya}. Education and perceptions: Evolution in Turkey. \textit{Science and Utopia Magazine}, \textit{Issue: 278} (2017).} && \\
\textbf{[3]} &\; \text{\textbf{Isin Altinkaya}. The evolving brain. \textit{Atheist Magazine}, \textit{Issue: 16} (2016).} && \\
\textbf{[2]} &\; \text{\textbf{Isin Altinkaya}. Evolutionary biology and LGBTI+ in nature. \textit{Atheist Magazine}, \textit{Issue: 14} (2016).} && \\
\textbf{[1]} &\; \text{\textbf{Isin Altinkaya}. Understanding evolution through the human body. \textit{Atheist Magazine}, \textit{Issue: 13} (2016).} && \\
\end{flalign*}
\vspace{-3em}


%\textbf{[1]} \hspace{0.42cm} Atheist Magazine, Issue: 12, Atheist thinking under the light of evolution. \\
%\smallskip

\section{Open Source Projects and Contributions}
\textbf{-}   \href{https://github.com/isinaltinkaya/ngsAMOVA}{ngsAMOVA}:A tool designed to perform AMOVA, $d_{xy}$, Neighbor-Joining, and IBD segment detection from NGS data in a probabilistic framework (see Publications/Manuscripts in Preparation). (2022 - present).  \\
\smallskip
\textbf{-}  \href{https://github.com/isinaltinkaya/gptchatteR}{gptchatteR}: An experimental R package wrapper for interacting with OpenAI LLM models in R console, providing a framework for using AI language models to help learning and conducting data analysis, data visualization and wrangling directly from R console. (2023). \\
\smallskip

\textbf{-}  \href{https://github.com/isinaltinkaya/google-slides_scripts}{google-slides\_scripts}: Custom scripts providing various additional functionalities to Google Slides. (2022). \\
\smallskip

\textbf{-}  \href{https://github.com/isinaltinkaya/vcfgl}{vcfgl}: A tool for fast and efficient simulation of genotype likelihoods (see Publications/5). (2022). \\
\smallskip

\textbf{-}  \href{https://github.com/ANGSD/angsd}{ANGSD}: Actively contributed to the development and maintenance of the ANGSD software (2021-2024).\\
\smallskip

\textbf{-}  \href{https://github.com/isinaltinkaya/bash-functions}{bash-functions}: Custom bash functions designed to perform various tasks, including runda (environment-safe wrapper for conda), treemd (format and print the tree structure of a git repository in MD format) (2021).
\smallskip

\textbf{-}  \href{https://github.com/isinaltinkaya/i3pomodoror}{i3pomodoror}: i3 window manager wrapper for the R library 'pomodoror' (2020). \\
\smallskip

\textbf{-}  \href{https://github.com/isinaltinkaya/PopulasyonBuyumesi}{PopulasyonBuyumesi}: An RShiny application for teaching population ecology through including population growth simulations and interactive plots. The RShiny app (in Turkish) is  available at \href{ PopulasyonBuyumesiSimulasyonlari.isinaltinkaya.com/}{isinaltinkaya.shinyapps.io/PopulasyonBuyumesiSimulasyonlari}, (2020).\\
\smallskip

\textbf{-}  \href{https://github.com/isinaltinkaya/cheatsheets}{cheatsheets}: Custom Markdown tutorials containing custom functions, examples, and tips for using various tools and programming languages effectively, such as Bash, Git, Jupyter, Make, Conda/Bioconda and R (2020). \\
\smallskip

\textbf{-}  \href{https://github.com/SamueleSoraggi/HMMploidy}{HMMploidy}: A tool to calculate ploidy levels from genotype likelihoods and coverage using Hidden Markov Models. Contributed to the development of the open source software as a part of an academic project (see Publications/2), (2020).\\
\smallskip

\textbf{-}  \href{https://github.com/isinaltinkaya/i3gcalendar}{i3gcalendar}: Google Calendar i3blocks integration for i3 window manager (2018). \\
\smallskip

%-----------SCHOLARSHIPS & AWARDS---------------
\section{Scholarships \& Awards}
\setlength{\abovedisplayskip}{0pt}
\setlength{\belowdisplayskip}{0pt}
\setlength{\abovedisplayshortskip}{0pt}
\setlength{\belowdisplayshortskip}{0pt}
\begin{flalign*}
\text{\textbf{[5]}} &\; \textbf{Erasmus+ Traineeship Grant}, \text{Imperial College London (2019)} && \\
\text{\textbf{[3, 4]}} &\; \textbf{Hacktoberfest Open-Source Contribution Award}, \text{DigitalOcean (2018 and 2019)} && \\
&\; \textit{Awarded for significant contributions to open-source projects.} && \\
\text{\textbf{[1, 2]}} &\; \textbf{Next-Generation Humanist Leaders Scholarship}, \text{Atheist Alliance International (2016 and 2017)} && \\
&\; \textit{Granted to support the development of future humanist leaders in underdeveloped or distressed countries.} &&
\end{flalign*} \vspace{-2em}

%-----------SOCIETY MEMBERSHIPS-----------------
%Society Memberships
\section{Memberships \& Affiliations}
\textbf{-} Society for Molecular Biology and Evolution, \textit{2023 - present} \\
\textbf{-} Society for the Study of Evolution, \textit{2021 - present} \\
\textbf{-} Free Software Foundation, \textit{2020 - present} \\
\textbf{-} Ecology and Evolutionary Biology Society of Turkey, \textit{2017 - present} \\

 
%-----------LANGUAGE-----------------

\section{Languages}
    \textbf{-}  \textbf{Turkish:} Native \\
    \textbf{-}  \textbf{English:} Native-like fluency (ILR level R-$5$, S-$4+$, L-$5$, W-$5$) \\
    \textbf{-}  \textbf{German:} Elementary proficiency (ILR level R-$1$, S-$0+$, L-$0$, W-$0+$) \\
    \textbf{-}  \textbf{Danish:} Elementary proficiency (ILR level R-$1$, S-$0+$, L-$0+$, W-$1$) \\

%-----------REFERENCES-----------------
\section{References}

\textbf{-} \textbf{\href{https://scholar.google.com/citations?user=PySbfcEAAAAJ&hl=en&oi=ao}{Rasmus Nielsen}} \orcidlink{0000-0003-0513-6591} $\sbullet[.5]$ rasmus\_nielsen@berkeley.edu \\
University of California, Berkeley, Department of Integrative Biology and Department of Statistics \\
University of Copenhagen, Faculty of Health and Medical Sciences, Globe Institute, Section for GeoGenetics \\
%ORCID: \href{https://orcid.org/0000-0003-0513-6591}{0000-0003-0513-6591} \\

\textbf{-} \textbf{\href{https://scholar.google.com/citations?hl=en&user=-YNWF4AAAAAJ}{Thorfinn Sand Korneliussen}} \orcidlink{0000-0001-7576-5380}  $\sbullet[.5]$ tskorneliussen@sund.ku.dk \\
University of Copenhagen, Faculty of Health and Medical Sciences, Globe Institute, Section for GeoGenetics \\
%ORCID: \href{https://orcid.org/0000-0001-7576-5380}{0000-0001-7576-5380} \\

\textbf{-} \textbf{\href{https://scholar.google.com/citations?hl=en&user=rQhiTmYAAAAJ}{Matteo Fumagalli}} \orcidlink{0000-0002-4084-2953}  $\sbullet[.5]$  m.fumagalli@qmul.ac.uk  \\
School of Biological and Behavioural Sciences, Queen Mary University of London, United Kingdom \\
%ORCID: \href{https://orcid.org/0000-0002-4084-2953}{0000-0002-4084-2953} \\


\end{document}  